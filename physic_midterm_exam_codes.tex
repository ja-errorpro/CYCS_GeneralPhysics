\documentclass[UTF8,a4paper,12pt]{article}
\usepackage[fontset=none]{ctex}
\usepackage[colorlinks,linkcolor=blue]{hyperref}
\usepackage[marginpar=2cm]{geometry}
\usepackage{wallpaper}
\usepackage{tcolorbox}
\usepackage{enumitem}
\usepackage{amssymb}
\usepackage{listings}
\usepackage{color}
\usepackage{hyperref}
\usepackage{xeCJK}

\lstset{
    frame=tb,
    language=python,
    basicstyle=\ttfamily,
    numbers=none,
    numberstyle=\tiny\color{blue},
    keywordstyle=\color{red},
    stringstyle=\color{green},
    commentstyle=\color{pink},
    aboveskip=3pt,
    belowskip=3pt,
    showstringspaces=false,
    columns=flexible,
    breaklines=true,
    breakatwhitespace=true,
    tabsize=4
}

\setmainfont{Times New Roman}
\xeCJKsetup{AutoFakeBold=true}
\setCJKmainfont{標楷體}
\setCJKmonofont{標楷體}

\title{\fontsize{18pt}{\baselineskip} \textbf{普通物理期中(電腦模擬)}}
\author{\fontsize{16pt}{\baselineskip}11127137 黃乙家}
\date{}

\begin{document}
\maketitle

\fontsize{12pt}{\baselineskip}

所有模擬程式碼皆放在Github上,

連結:\url{https://github.com/ja-errorpro/GeneralPhysicsMidtermExam}

\section{低軌衛星題}

\begin{lstlisting}
    # 1_B.py 計算同步衛星高度
    import math
    R = 6.4e6
    T = 86400
    g = 9.8
    pi = math.pi
    H = (g*(R**2)*(T**2)/(4*(pi**2)))**(1/3) - R
    print(H, "m, or ", H/1000, "km")
\end{lstlisting}

\newpage

\begin{lstlisting}
    # 1_C.py 計算低空衛星周期
    import math
    R = 6.4e6
    T = 86400
    g = 9.8
    pi = math.pi
    H_B = (g*(R**2)*(T**2)/(4*(pi**2)))**(1/3) - R
    H_C = 5e5
    T_C = math.sqrt(((R+H_C)**3)/((R+H_B)**3)*(T**2))
    print(T_C, "s")
\end{lstlisting}

\newpage

\begin{lstlisting}
    # 1_D.py 計算衛星覆蓋面積
    import math
    R = 6400
    pi = math.pi
    H_B = 35940
    H_C = 500
    S_B = 2*pi*(R**2)*(1-R/(R+H_B))
    S_C = 2*pi*(R**2)*(1-R/(R+H_C))
    print("S_B: ", S_B, "km^2")
    print("S_C: ", S_C, "km^2")
\end{lstlisting}

\newpage

\section{化學碰撞學說題}

\begin{lstlisting}
    # 2_B.py 畫出 Activation Energy 與 m 的關係圖
    import matplotlib.pyplot as plt
    import numpy as np
    
    T = 323
    k = 1.38e-23
    m_A = np.linspace(1,100,100) # 設 m_A = 1~100 kg
    m_B = np.linspace(50,100,100) # 設 m_B = 50~100 kg
    E_act = 3*k*T - (3*k*T * (m_A + 50 - 2 * (m_A * 50)**0.5) ) / ( 2 * (m_A + 50) ) # 計算 Activation Energy
    plt.plot(m_A,E_act, label = 'm_A')
    plt.plot(m_B,E_act, label = 'm_B')
    plt.plot(m_A + m_B,E_act, label = 'm_A + m_B')
    plt.legend()
    plt.xlabel('m (kg)')
    plt.ylabel('Activation Energy (J)')
    plt.show()
\end{lstlisting}

\newpage

\section{馬尾題}

\begin{lstlisting}
    # 3_C_1.py 求角度微分方程解
    import numpy as np
    import matplotlib.pyplot as plt
    from scipy.integrate import odeint
    
    g = 9.8
    l = 1
    def diff(y, t):
        omega, theta = y
        return np.array([-(g/l)*np.sin(theta), omega])
    t = np.linspace(0, 10, 1000)
    theta_0 = 50 / 180 * np.pi
    ret = odeint(diff, [0, theta_0 ], t)
    
    plt.plot(t, ret[:, 0])
    plt.plot(t, ret[:, 1])
    plt.show()
\end{lstlisting}

\newpage

\begin{lstlisting}
    # 3_C_2.py 模擬週期
    import numpy as np
    import matplotlib.pyplot as plt
    from scipy import special
    
    theta_0 = np.linspace(0, np.pi, 100)
    L = 1
    g = 9.8
    omega_0 = np.sqrt(g/L)
    T_0 = 2*np.pi/omega_0
    T = 4*np.sqrt(L/g)*special.ellipk(np.sin(theta_0/2))
    plt.plot(theta_0, T/T_0)
    plt.xlabel(r'$\theta_0$')
    plt.ylabel(r'$T/T_0$')
    plt.show()
    
\end{lstlisting}

\newpage


\begin{lstlisting}
    # 3_D_1.py 推導微分方程
    from sympy import *
    from sympy import Derivative as D
    
    var("x1 x2 y1 y2 L1 L2 m1 m2 dtheta1 dtheta2 ddtheta1 ddtheta2 t g tmp")
    
    var("theta1 theta2", cls=Function)
    
    
    sublist = [
        (D(theta1(t), t, t), ddtheta1),
        (D(theta1(t), t), dtheta1),
        (D(theta2(t), t, t), ddtheta2),
        (D(theta2(t), t), dtheta2),
        (theta1(t), theta1()),
        (theta2(t), theta2())
    ]
    
    x1 = L1 * sin(theta1(t))
    y1 = -L1 * cos(theta1(t))
    x2 = x1 + L2 * sin(theta2(t))
    y2 = y1 - L2 * cos(theta2(t))
    
    vx1 = diff(x1, t)
    vy1 = diff(y1, t)
    vx2 = diff(x2, t)
    vy2 = diff(y2, t)
    
    L = m1/2 * (vx1**2 + vy1**2) + m2/2 * (vx2**2 + vy2**2) - m1 * g * y1 - m2 * g * y2
    
    def lagrange(L, v):
        dv = D(v(t),t)
        a = L.subs(dv, tmp).diff(tmp).subs(tmp, dv)
        b = L.subs(dv, tmp)
        b = b.subs(v(t),v())
        b = b.diff(v())
        b = b.subs(v(), v(t))
        b = b.subs(tmp, dv)
        c = a.diff(t) - b
        c = c.subs(sublist)
        c = trigsimp(simplify(c))
        c = collect(c, [theta1(),theta2(),dtheta1,dtheta2,ddtheta1,ddtheta2])
        return c
    
    eq1 = lagrange(L, theta1)
    eq2 = lagrange(L, theta2)
    
    print("eq1 = ", eq1)
    print("eq2 = ", eq2)
\end{lstlisting}

\newpage

\begin{lstlisting}
    # 3_D_2.py 模擬動畫
    import matplotlib
    matplotlib.use('WXAgg')
    import matplotlib.pyplot as plt
    
    import numpy as np
    from scipy.integrate import odeint
    from math import *
    import wx
    
    
    g = 9.8
    class DoublePendulum(object):
        def __init__(self,m1,m2,L1,L2):
            self.m1, self.m2 = m1, m2
            self.L1, self.L2 = L1, L2
            self.init_stat = np.array([0.0, 0.0, 0.0, 0.0])
    
        def equations(self,w,t):
            m1, m2, L1, L2 = self.m1, self.m2, self.L1, self.L2
            theta1, theta2, v1, v2 = w
            dth1 = v1
            dth2 = v2
    
            eq1a = (m1+m2)*L1*L1
            eq1b = m2*L1*L2*cos(theta1-theta2)
            eq1c = L1*(m2*L2*dth2*dth2*sin(theta1-theta2) + (m1+m2)*g*sin(theta1))
    
            eq2a = L1*m2*L2*cos(theta1-theta2)
            eq2b = L2*L2*m2
            eq2c = m2*L2*(-L1*dth1*dth1*sin(theta1-theta2) + g*sin(theta2))
    
            dv1, dv2 = np.linalg.solve([[eq1a, eq1b], [eq2a, eq2b]], [-eq1c, -eq2c])
    
            return np.array([dth1, dth2, dv1, dv2])
    
    def double_pendulum_odeint(pendulum, l, r, step):
        t = np.arange(l,r,step)
        trk = odeint(pendulum.equations, pendulum.init_stat, t)
        theta1, theta2 = trk[:,0], trk[:,1]
        L1 = pendulum.L1
        L2 = pendulum.L2
        x1 = L1*np.sin(theta1)
        y1 = -L1*np.cos(theta1)
        x2 = x1 + L2*np.sin(theta2)
        y2 = y1 - L2*np.cos(theta2)
        pendulum.init_stat = trk[-1,:].copy()
        return [x1, y1, x2, y2]
    
    fig = plt.figure(figsize=(6,6))
    
    line1, = plt.plot([0,0],[0,0],"-o")
    line2, = plt.plot([0,0],[0,0],"-o")
    plt.axis("equal")
    plt.xlim(-5,5)
    plt.ylim(-5,5)
    
    print('模擬雙擺運動(若直接按下Enter則使用預設值):')
    
    m1 = input('請輸入m1質量[1.0]:')
    if m1 == '':
        m1 = 1.0
    m2 = input('請輸入m2質量[1.0]:')
    if m2 == '':
        m2 = 1.0
    L1 = input('請輸入L1長度[1.0]:')
    if L1 == '':
        L1 = 1.0
    L2 = input('請輸入L2長度[1.0]:')
    if L2 == '':
        L2 = 1.0
    
    pendulum = DoublePendulum(m1, m2, L1, L2)
    theta1 = input('請輸入初始theta1角度(徑度)[1.0]:')
    if theta1 == '':
        theta1 = 1.0
    theta2 = input('請輸入初始theta2角度(徑度)[1.0]:')
    if theta2 == '':
        theta2 = 1.0
    pendulum.init_stat[:2] = theta1, theta2
    
    x1,y1,x2,y2 = double_pendulum_odeint(pendulum, 0, 30, 0.02)
    plt.plot(x1,y1,label="m_1")
    plt.plot(x2,y2,label="m_2")
    
    plt.title("m1 = %s, m2 = %s, L1 = %s, L2 = %s, theta1 = %s, theta2 = %s" % (m1, m2, L1, L2, theta1, theta2))
    
    idx = 0
    
    def update_line(event):
        global x1,y1,x2,y2,idx
        if idx == len(x1):
            idx = 0
            x1, y1, x2, y2 = double_pendulum_odeint(pendulum, 0, 30, 0.02)
        line1.set_xdata([0,x1[idx]])
        line1.set_ydata([0,y1[idx]])
        line2.set_xdata([x1[idx],x2[idx]])
        line2.set_ydata([y1[idx],y2[idx]])
        fig.canvas.draw()
        idx += 1
    
    id = wx.ID_ANY
    actor = fig.canvas.manager.frame
    actor.Bind(wx.EVT_TIMER, update_line, id=id)
    timer = wx.Timer(actor, id)
    timer.Start(1)
    
    plt.legend()
    
    plt.show()
\end{lstlisting}

\end{document}